\documentclass[a4paper,10pt,oneside]{article}
\setlength{\columnsep}{15pt}    %兩欄模式的間距
\setlength{\columnseprule}{0pt}

\newcommand{\NumberOfColumn}{4} % 每頁有幾欄
\newcommand{\UniversityName}{National Tsing Hua University} % 學校名稱
\newcommand{\TeamName}{Angry Crow Takes Flight!} % 隊名

\usepackage[landscape]{geometry}
\usepackage{amsthm}								%定義,例題
\usepackage{amssymb}
\usepackage{fontspec}								%設定字體
\usepackage{color}
\usepackage[x11names]{xcolor}
\usepackage{xeCJK}								%xeCJK
\usepackage{listings}								%顯示code用的
%\usepackage[Glenn]{fncychap}						%排版,頁面模板
\usepackage{fancyhdr}								%設定頁首頁尾
\usepackage{graphicx}								%Graphic
\usepackage{enumerate}
\usepackage{titlesec}
\usepackage{amsmath}
\usepackage{pdfpages}
\usepackage{multicol}
\usepackage{fancyhdr}
%\usepackage[T1]{fontenc}
\usepackage{amsmath, courier, listings, fancyhdr, graphicx}

%\topmargin=0pt
%\headsep=5pt
\textheight=530pt
%\footskip=0pt
\voffset=-20pt
\textwidth=800pt
%\marginparsep=0pt
%\marginparwidth=0pt
%\marginparpush=0pt
%\oddsidemargin=0pt
%\evensidemargin=0pt
\hoffset=-100pt

%\setmainfont{Consolas}				%主要字型
\setCJKmainfont{msjh.ttc}			%中文字型
%\setmainfont{Linux Libertine G}
\setmonofont{consola.ttf}
%\setmainfont{sourcecodepro}
\XeTeXlinebreaklocale "zh"						%中文自動換行
\XeTeXlinebreakskip = 0pt plus 1pt				%設定段落之間的距離
\setcounter{secnumdepth}{3}						%目錄顯示第三層

\makeatletter
\lst@CCPutMacro\lst@ProcessOther {"2D}{\lst@ttfamily{-{}}{-{}}}
\@empty\z@\@empty
\makeatother
\lstset{											% Code顯示
basicstyle=\scriptsize\ttfamily, 						% the size of the fonts that are used for the code
numbers=left,										% where to put the line-numbers
numberstyle=\tiny,						% the size of the fonts that are used for the line-numbers
stepnumber=1,										% the step between two line-numbers. If it's 1, each line  will be numbered
numbersep=5pt,										% how far the line-numbers are from the code
backgroundcolor=\color{white},					% choose the background color. You must add \usepackage{color}
showspaces=false,									% show spaces adding particular underscores
showstringspaces=false,							% underline spaces within strings
showtabs=false,									% show tabs within strings adding particular underscores
frame=false,											% adds a frame around the code
tabsize=2,											% sets default tabsize to 2 spaces
captionpos=b,										% sets the caption-position to bottom
breaklines=true,									% sets automatic line breaking
breakatwhitespace=false,							% sets if automatic breaks should only happen at whitespace
escapeinside={\%*}{*)},							% if you want to add a comment within your code
morekeywords={*},									% if you want to add more keywords to the set
keywordstyle=\bfseries\color{Blue1},
commentstyle=\itshape\color{Red4},
stringstyle=\itshape\color{Green4},
}


\newcommand{\includecpp}[2]{
  \subsection{#1}
    \lstinputlisting[language=C++]{#2}
}

\newcommand{\includepy}[2]{
  \subsection{#1}
    \lstinputlisting[language=python]{#2}
}

\newcommand{\includetex}[2]{
  \subsection{#1}
    \input{#2}
}


\begin{document}
  \begin{multicols}{\NumberOfColumn}
  \pagestyle{fancy}
  
  \fancyfoot{}
  \fancyhead[L]{\UniversityName{} - \TeamName}
  \fancyhead[R]{\thepage}
  
  \renewcommand{\headrulewidth}{0.4pt}
  \renewcommand{\contentsname}{Contents}

   
  \scriptsize
  \section{Computational\_Geometry}
  \includecpp{delaunay}{./Computational_Geometry/delaunay.cpp}
  \includecpp{Geometry}{./Computational_Geometry/Geometry.cpp}
  \includecpp{SmallestCircle}{./Computational_Geometry/SmallestCircle.cpp}
  \includecpp{最近點對}{./Computational_Geometry/最近點對.cpp}
\section{Data\_Structure}
  \includecpp{CDQ\_DP}{./Data_Structure/CDQ_DP.cpp}
  \includecpp{DLX}{./Data_Structure/DLX.cpp}
  \includecpp{Dynamic\_KD\_tree}{./Data_Structure/Dynamic_KD_tree.cpp}
  \includecpp{kd\_tree\_replace\_segment\_tree}{./Data_Structure/kd_tree_replace_segment_tree.cpp}
  \includecpp{reference\_point}{./Data_Structure/reference_point.cpp}
  \includecpp{skew\_heap}{./Data_Structure/skew_heap.cpp}
  \includecpp{undo\_disjoint\_set}{./Data_Structure/undo_disjoint_set.cpp}
  \includecpp{整體二分}{./Data_Structure/整體二分.cpp}
\section{Flow}
  \includecpp{dinic}{./Flow/dinic.cpp}
  \includecpp{Gomory\_Hu}{./Flow/Gomory_Hu.cpp}
  \includecpp{ISAP\_with\_cut}{./Flow/ISAP_with_cut.cpp}
  \includecpp{MinCostMaxFlow}{./Flow/MinCostMaxFlow.cpp}
\section{Graph}
  \includecpp{Augmenting\_Path}{./Graph/Augmenting_Path.cpp}
  \includecpp{Augmenting\_Path\_multiple}{./Graph/Augmenting_Path_multiple.cpp}
  \includecpp{blossom\_matching}{./Graph/blossom_matching.cpp}
  \includecpp{BronKerbosch}{./Graph/BronKerbosch.cpp}
  \includecpp{graphISO}{./Graph/graphISO.cpp}
  \includecpp{KM}{./Graph/KM.cpp}
  \includecpp{MaximumClique}{./Graph/MaximumClique.cpp}
  \includecpp{MinimumMeanCycle}{./Graph/MinimumMeanCycle.cpp}
  \includecpp{Rectilinear\_MST}{./Graph/Rectilinear_MST.cpp}
  \includecpp{treeISO}{./Graph/treeISO.cpp}
  \includecpp{一般圖最小權完美匹配}{./Graph/一般圖最小權完美匹配.cpp}
  \includecpp{全局最小割}{./Graph/全局最小割.cpp}
  \includecpp{弦圖完美消除序列}{./Graph/弦圖完美消除序列.cpp}
  \includecpp{最小斯坦納樹DP}{./Graph/最小斯坦納樹DP.cpp}
  \includecpp{最小樹形圖\_朱劉}{./Graph/最小樹形圖_朱劉.cpp}
  \includecpp{穩定婚姻模板}{./Graph/穩定婚姻模板.cpp}
\section{Linear\_Programming}
  \includecpp{simplex}{./Linear_Programming/simplex.cpp}
\section{Number\_Theory}
  \includecpp{basic}{./Number_Theory/basic.cpp}
  \includecpp{bit\_set}{./Number_Theory/bit_set.cpp}
  \includecpp{cantor\_expansion}{./Number_Theory/cantor_expansion.cpp}
  \includecpp{FFT}{./Number_Theory/FFT.cpp}
  \includecpp{find\_real\_root}{./Number_Theory/find_real_root.cpp}
  \includecpp{FWT}{./Number_Theory/FWT.cpp}
  \includecpp{LinearCongruence}{./Number_Theory/LinearCongruence.cpp}
  \includecpp{Lucas}{./Number_Theory/Lucas.cpp}
  \includecpp{Matrix}{./Number_Theory/Matrix.cpp}
  \includecpp{MillerRobin}{./Number_Theory/MillerRobin.cpp}
  \includecpp{NTT}{./Number_Theory/NTT.cpp}
  \includecpp{Simpson}{./Number_Theory/Simpson.cpp}
  \includecpp{外星模運算}{./Number_Theory/外星模運算.cpp}
  \includecpp{數位統計}{./Number_Theory/數位統計.cpp}
  \includecpp{質因數分解}{./Number_Theory/質因數分解.cpp}
\section{String}
  \includecpp{AC自動機}{./String/AC自動機.cpp}
  \includecpp{hash}{./String/hash.cpp}
  \includecpp{KMP}{./String/KMP.cpp}
  \includecpp{manacher}{./String/manacher.cpp}
  \includecpp{minimal\_string\_rotation}{./String/minimal_string_rotation.cpp}
  \includecpp{reverseBWT}{./String/reverseBWT.cpp}
  \includecpp{suffix\_array\_lcp}{./String/suffix_array_lcp.cpp}
  \includecpp{Z}{./String/Z.cpp}
\section{Tarjan}
  \includecpp{dominator\_tree}{./Tarjan/dominator_tree.cpp}
  \includecpp{tnfshb017\_2\_sat}{./Tarjan/tnfshb017_2_sat.cpp}
  \includecpp{橋連通分量}{./Tarjan/橋連通分量.cpp}
  \includecpp{雙連通分量\&割點}{./Tarjan/雙連通分量&割點.cpp}
\section{Tree\_problem}
  \includecpp{HeavyLight}{./Tree_problem/HeavyLight.cpp}
  \includecpp{LCA}{./Tree_problem/LCA.cpp}
  \includecpp{link\_cut\_tree}{./Tree_problem/link_cut_tree.cpp}
  \includecpp{POJ\_tree}{./Tree_problem/POJ_tree.cpp}
\section{default}
  \includecpp{debug}{./default/debug.cpp}
  \includecpp{ext}{./default/ext.cpp}
  \includecpp{IncStack}{./default/IncStack.cpp}
  \includecpp{input}{./default/input.cpp}
\section{language}
  \includecpp{CNF}{./language/CNF.cpp}
\section{other}
  \includecpp{WhatDay}{./other/WhatDay.cpp}
  \includecpp{上下最大正方形}{./other/上下最大正方形.cpp}
  \includecpp{最大矩形}{./other/最大矩形.cpp}
\section{zformula}
  \includetex{formula}{./zformula/formula.tex}
  \includetex{java}{./zformula/java.tex}
\section{Интернационал}
  \includecpp{ganadoQuote}{./Интернационал/ganadoQuote.cpp}
  \includecpp{Интернационал}{./Интернационал/Интернационал.cpp}
  \includecpp{保佑}{./Интернационал/保佑.cpp}

  \clearpage
  \end{multicols}
  \newpage
  \begin{multicols}{\NumberOfColumn}
  \enlargethispage*{\baselineskip}
  \begin{center}
    \Huge\textsc{ACM ICPC Team Reference - \TeamName}
    \vspace{0.35cm}    
  \end{center}
  \tableofcontents
  \end{multicols}
  \clearpage
\end{document}
